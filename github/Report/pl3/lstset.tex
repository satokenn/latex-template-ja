
% \usepackage{geometry}
\usepackage[dvipdfmx]{graphicx}
\usepackage{ascmac}
\usepackage{amsmath}
\usepackage{amssymb}
\usepackage{minted}
% \usepackage{jlistings}
% \usepackage[utf8]{inputenc}
% \usepackage{listingsutf8}
\usepackage{float}
\usepackage{fix-cm}
\usepackage{svg}

\usepackage{hyperref}
% for hyperref
\usepackage{pxjahyper}
\hypersetup{
    dvipdfmx,
	colorlinks=false, % リンクに色をつけない設定
	bookmarks=true, % 以下ブックマークに関する設定
	bookmarksnumbered=true,
	pdfborder={0 0 0},
	bookmarkstype=toc
}

% \lstset{
%   basicstyle={\ttfamily},
%   identifierstyle={\small},
%   inputencoding=auto,
%   commentstyle={\small\sffamily},
%   keywordstyle={\small\bfseries},
%   ndkeywordstyle={\small},
%   stringstyle={\small\ttfamily},
%   frame={tb},
%   breaklines=true,
%   columns=[l]{fullflexible},
%   numbers=left,
%   % xrightmargin=0zw,
%   % xleftmargin=3zw,
%   numberstyle={\scriptsize},
%   firstnumber=auto,
%   stepnumber=1,
%   numbersep=5pt,
%   lineskip=1ex
% }
\setminted{
  mathescape,              % 数式モードへのエスケープを許可 (必要なら)
  % basicstyle やフォント関連
  fontsize=\small,         % 全体のフォントサイズ (listings の \small に合わせる試み)
                           % \ttfamily は minted のデフォルトに近いが、日本語対応の等幅フォントを
                           % LaTeX 側で \ normaalfont や \ttdefault に設定しておくのが理想
  % frame
  frame=lines,             % 上下に線を引く frame=tb に近いものとして lines (上下左右に線)
  framesep=2mm,            % 枠線とコードの間隔 (調整が必要)
  % breaklines
  breaklines=true,         % 自動折り返し
  % numbers
  linenos=true,            % 行番号を左に表示
  firstnumber=auto,        % 行番号を開始行に合わせる
  numbersep=5pt,           % 行番号とコードの間隔
  % stepnumber=1,          % linenos=true で通常1ステップ
  % highlight と Pygments スタイル
  % minted では Pygments のスタイルを使います。
  % style=friendly のようにスタイル名を指定できます。
  % デフォルトのスタイルでキーワードが太字になるか確認。
  % commentstyle や keywordstyle の LaTeX コマンド直接指定はできません。
  % Pygments のスタイルでこれらがどのように見えるか確認し、
  % 必要ならカスタムスタイルを作るか、別の既存スタイルを選びます。
}