
% \usepackage{geometry}
\usepackage[dvipdfmx]{graphicx}
\usepackage{ascmac}
\usepackage{amsmath}
\usepackage{amssymb}
\usepackage{minted}
\usepackage{listings}
% \usepackage[utf8]{inputenc}
% \usepackage{listingsutf8}
\usepackage{newfloat} % newfloat パッケージを読み込む
\usepackage{caption}
\usepackage{float}
\usepackage{fix-cm}
\usepackage{svg}
\usepackage{enumitem}

\usepackage{hyperref}
% for hyperref
\usepackage{pxjahyper}
\hypersetup{
    dvipdfmx,
	colorlinks=false, % リンクに色をつけない設定
	bookmarks=true, % 以下ブックマークに関する設定
	bookmarksnumbered=true,
	pdfborder={0 0 0},
	bookmarkstype=toc
}

% \lstset{
%   basicstyle={\ttfamily},
%   identifierstyle={\small},
%   inputencoding=auto,
%   commentstyle={\small\sffamily},
%   keywordstyle={\small\bfseries},
%   ndkeywordstyle={\small},
%   stringstyle={\small\ttfamily},
%   frame={tb},
%   breaklines=true,
%   columns=[l]{fullflexible},
%   numbers=left,
%   % xrightmargin=0zw,
%   % xleftmargin=3zw,
%   numberstyle={\scriptsize},
%   firstnumber=auto,
%   stepnumber=1,
%   numbersep=5pt,
%   lineskip=1ex
% }

% 新しい浮動体「listing」を定義
\DeclareFloatingEnvironment[
    fileext=lol,       % List of Listings ファイルの拡張子 (List of Listings を作成する場合)
    name=Listing,      % キャプションの接頭辞 (例: "Listing 1")
    placement={!htbp}, % フロートの配置オプション (お好みで調整してください)
    within=section     % 番号付けをセクションごとにする場合 (例: Listing 1.1) (任意、不要なら削除)
]{program}

\setminted{
  mathescape,              % 数式モードへのエスケープを許可 (必要なら)
  % basicstyle やフォント関連
  fontsize=\small,         % 全体のフォントサイズ (listings の \small に合わせる試み)
                           % \ttfamily は minted のデフォルトに近いが、日本語対応の等幅フォントを
                           % LaTeX 側で \ normaalfont や \ttdefault に設定しておくのが理想
  % frame
  frame=lines,             % 上下に線を引く frame=tb に近いものとして lines (上下左右に線)
  framesep=2mm,            % 枠線とコードの間隔 (調整が必要)
  % breaklines
  breaklines=true,         % 自動折り返し
  % numbers
  linenos=true,            % 行番号を左に表示
  firstnumber=auto,        % 行番号を開始行に合わせる
  numbersep=5pt,           % 行番号とコードの間隔
  % stepnumber=1,          % linenos=true で通常1ステップ
  % highlight と Pygments スタイル
  % minted では Pygments のスタイルを使います。
  % style=friendly のようにスタイル名を指定できます。
  % デフォルトのスタイルでキーワードが太字になるか確認。
  % commentstyle や keywordstyle の LaTeX コマンド直接指定はできません。
  % Pygments のスタイルでこれらがどのように見えるか確認し、
  % 必要ならカスタムスタイルを作るか、別の既存スタイルを選びます。
}
\title{title}
\author{情報学群 \\ 1270328 佐藤謙成}
\date{\today}

\begin{document}
\maketitle

\begin{abstract}
    
\end{abstract}
\section{全体の目的}
\section{第六回の目的}
    第六回では,地球に到達する光を分析することにより,恒星の動きを知ることを目的とする.恒星がどの程度の速度で地球から遠ざかっているのかを算出し,観測されたスペクトルを描画する.また,単一の恒星についての分析と複数の恒星についてどのよう分析を行い,~~する.
\section{第七回の目的}
\section{第八回の目的}
\section{第九回の目的}
\section{第十回の目的}

\section{第六回の方法}
    \lstinputlisting[caption = データの読み込み,
        label={lst:exp3_6_load},
        linerange={1-2}]{code/Exp3_6_Matlab.m}
    \ref{lst:exp3_6_load}では,まず starData というデータファイルを読み込み,\lstinline|size(spectra,1)| で各スペクトルに含まれる観測点の数を取得する.

    次に,恒星 HD94028 のスペクトルを抽出する.それが以下のコード \ref{lst:exp3_6_s}である.
    \lstinputlisting[caption = 恒星 HD94028 のスペクトル表示
        label={lst:exp3_6_s},
        linerange={14-19}]{code/Exp3_6_Matlab.m}
    ここでは,両軸対数スケールで出力するために関数 loglog を用いている.また,恒星 HD94028 のスペクトルはベクトル s に格納する.

    ベクトル s に格納された値を用いて水素アルファ線の波長を求める.コード \ref{lst:exp3_6_alpha} では, s の最小値が水素アルファ線であることを利用して \lstinline|min(s)| で求めている.関数 min はここでは 2つの値を出力することができる.一つ目の値が水素アルファ線の値で二つ目の値がそのインデックスとなる.

    \lstinputlisting[caption= 水素アルファ線の特定,
    label={lst:exp3_6_alpha},
    linerange={20-23}]{code/Exp3_6_Matlab.m}

    特定した水素アルファ線に対して点を追加するためのコードがコード\ref{lst:exp3_6_hold}である. ここでは, \lstinline|hold on| を記述することで既存のグラフに点を追記することができ, \lstinline|hold off| で追記を終了することができる.今回は,水素アルファ線の値の部分にマーカーサイズが 8 の赤い正方形をプロットする.
    \lstinputlisting[caption = 既存グラフへの追記,
    label = {lst:exp3_6_hold},
    linerange={24-27}]{code/Exp3_6_Matlab.m}
    
    赤方偏移係数と星が地球から遠ざかる速度を求める.赤方偏移係数は係数を $z$ としたときに $z = \dfrac{\lambda Ha }{656.28} - 1$ で求めることができるため,これを実装する.速度に関しては,赤方偏移係数に光速の値をかけることで求めることができる.
    \lstinputlisting[caption = 赤方偏移係数と速度の計算,
    label={lst:exp3_6_spped},
    linerange={29-31}]{code/Exp3_6_Matlab.m}

    ここから,各星の水素アルファ線を求める.行列の各行に対して最小値とそのインデックスを計算する.これは \lstinline|min(spectra(:,:))| で求めることができる.
\end{document}
