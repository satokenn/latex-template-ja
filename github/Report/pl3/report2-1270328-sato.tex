\documentclass[dvipdfmx, titlepage, t]{jsarticle}
% コピペスタート
\documentclass[8pt, jfont=ipaexm, t]{beamer} % IPAex明朝
\usepackage{hyperref}
% 一般的によく使用されるパッケージ
\usepackage{caption}
\usepackage[utf8]{inputenc}
\usepackage{graphicx}
\usepackage{amsmath}
\usepackage{amsfonts}
\usepackage{amssymb}
\usepackage{tikz}
\usepackage{pgfplots}
\pgfplotsset{compat=1.18}   % pgfplots のバージョンを指定
\usepackage{listings}
\usepackage{array}

% スタイル設定
\usepackage{ifthen}
\usepackage[varg]{txfonts}
\usepackage{ragged2e}
\usepackage{svg}
\usepackage{xcolor}
\usepackage{url}
\usepackage{bm}

\usetikzlibrary{graphs}
\usetikzlibrary {arrows.meta}
\usetikzlibrary {bending}
\usetikzlibrary{arrows,shapes,automata,petri,positioning,calc}


% 和文用パッケージ(luatex用)
\usepackage{luatexja}
% \usepackage{luatexja-fontspec}
\usepackage{lmodern}
\usepackage[T1]{fontenc} % 必要に応じてフォントエンコーディングを指定


%カラーテーマの選択(省略可)
\usecolortheme{orchid}
%フォントテーマの選択(省略可)
\usefonttheme{professionalfonts}
%フレーム内のテーマの選択(省略可)
\useinnertheme{circles}
%フレーム外側のテーマの選択(省略可)
\useoutertheme{infolines}
%しおりの文字化け解消
% \AtBeginShipoutFirst{\special{pdf:tounicode EUC-UCS2}}

% \AtBeginShipoutFirst{\special{pdf:tounicode 90ms-RKSJ-UCS2}}
%ナビゲーションバー非表示
\setbeamertemplate{navigation symbols}{}

% タイトル色
\setbeamercolor{title}{fg=structure, bg=}

% フレームタイトル色
\setbeamercolor{frametitle}{fg=structure, bg=}

% caption に番号追加
\setbeamertemplate{caption}[numbered]
% caption 日本語
\renewcommand{\figurename}{図}
\renewcommand{\tablename}{表}

\usepackage[export]{adjustbox} % loads also graphicx


\usetheme[progressbar=frametitle, block=fill, numbering=fraction,]{metropolis}
% \usetheme{default}
            
% ブロックのスタイルをカスタマイズ
\setbeamertemplate{blocks}[rounded]
\setbeamercolor{block title}{bg=gray!30,fg=black} % ブロックのタイトルの背景とフォントの色
\setbeamercolor{block body}{bg=gray!10,fg=black} % ブロック本体の背景とフォントの色

\setbeamercolor{block title example}{bg=orange!30,fg=black} % 例のブロックのタイトルの背景とフォントの色
\setbeamercolor{block body example}{bg=orange!10,fg=black} % 例のブロック本体の背景とフォントの色

\setbeamercolor{block title alerted}{bg=red!30,fg=black} % アラートのブロックのタイトルの背景とフォントの色
\setbeamercolor{block body alerted}{bg=red!10,fg=black} % アラートのブロック本体の背景とフォントの色

\tikzset{set label/.style={fill=white,circle,inner sep=2}}

\def\radius{2}
\def\ratio{0.6}

\def\centerA{180:\ratio*\radius}
\def\circleA{(\centerA) circle [radius=\radius]}





%追加
\setbeamertemplate{footline}{%
  \hfill%
  \usebeamercolor[fg]{page number in head/foot}%
  \usebeamerfont{page number in head/foot}%
  \insertframenumber\,/\,\inserttotalframenumber\kern1em\vskip2pt%
}

%ソースコードに関する設定
\lstset{
    language={C++}, 
    basicstyle={\ttfamily},
    identifierstyle={\small},
    commentstyle={\smallitshape},
    keywordstyle={\small\bfseries},
    ndkeywordstyle={\small},
    stringstyle={\small\ttfamily},
    frame={tb},
    breaklines=true,
    columns=[l]{fullflexible},
    numbers=left,
    xrightmargin=0em,
    xleftmargin=3em,
    numberstyle={\scriptsize},
    stepnumber=1,
    numbersep=1em,
    lineskip=-0.5ex
}


\tikzset{
    place/.style={
        circle,
        thick,
        draw=black,
        fill=gray!50,
        minimum size=6mm,
    },
        state/.style={
        circle,
        thick,
        draw=blue!75,
        fill=blue!20,
        minimum size=6mm,
    },
}

\title{title}
\author{情報学群 \\ 1270328 佐藤謙成}
\date{\today}

\begin{document}
\maketitle

\begin{abstract}
    
\end{abstract}
\section{全体の目的}
\subsection{第六回の目的}
    第六回では,地球に到達する光を分析することにより,恒星の動きを知ることを目的とする.恒星がどの程度の速度で地球から遠ざかっているのかを算出し,観測されたスペクトルを描画する.また,単一の恒星についての分析と複数の恒星についてどのよう分析を行い,~~する.
\subsection{第七回の目的}
    第七回では,次回の眼球運動継続実験の準備としてMatLab及びPsychotoolboxを用いてプログラムを作成する.本課題では,被験者が左右に提示される顔画像の魅力を判断し,キー押しで反応する二者択一の選択課題を 2 試行行う実験環境を構築することである.
\subsection{第八回の目的}
\subsection{第九回の目的}
    第八回では,EyeLink II によって過去に測定された眼球運動実験のデータをプログラムを用いて解析することである.被験者が左右の顔画像のどちらかを選択する際に,選択した顔画像んいどの程度視線を向けていたかの確率を時間経過とともにグラフかすることを目的とする.
\subsection{第十回の目的}
第十回では,異なる負荷条件下で記録された筋電図データを解析し,各条件下での筋活動レベルを定量的に評価することである.

\section{方法}
\subsection{第六回の方法}
    \inputminted[linenos, firstline=1, lastline=2, frame=lines, fontsize=\small]{matlab}{code/Exp3_6_Matlab.m}
    \ref{lst:exp3_6_load}では,まず starData というデータファイルを読み込み,\mintinline{matlab}|size(spectra,1)| で各スペクトルに含まれる観測点の数を取得する.

    次に,恒星 HD94028 のスペクトルを抽出する.それが以下のコード \ref{lst:exp3_6_s}である.
    \inputminted[linenos, firstline=14, lastline=19, frame=lines, fontsize=\small]{matlab}{code/Exp3_6_Matlab.m}
    ここでは,両軸対数スケールで出力するために関数 loglog を用いている.また,恒星 HD94028 のスペクトルはベクトル s に格納する.

    ベクトル s に格納された値を用いて水素アルファ線の波長を求める.コード \ref{lst:exp3_6_alpha} では, s の最小値が水素アルファ線であることを利用して \mintinline{matlab}|min(s)| で求めている.関数 min はここでは 2つの値を出力することができる.一つ目の値が水素アルファ線の値で二つ目の値がそのインデックスとなる.

    \inputminted[linenos, firstline=20, lastline=23, frame=lines, fontsize=\small]{matlab}{code/Exp3_6_Matlab.m}

    特定した水素アルファ線に対して点を追加するためのコードがコード\ref{lst:exp3_6_hold}である. ここでは, \mintinline{matlab}|hold on| を記述することで既存のグラフに点を追記することができ, \mintinline{matlab}|hold off| で追記を終了することができる.今回は,水素アルファ線の値の部分にマーカーサイズが 8 の赤い正方形をプロットする.
    \inputminted[linenos, firstline=24, lastline=27, frame=lines, fontsize=\small]{matlab}{code/Exp3_6_Matlab.m}
    
    赤方偏移係数と星が地球から遠ざかる速度を求める.赤方偏移係数は係数を $z$ としたときに $z = \dfrac{\lambda Ha }{656.28} - 1$ で求めることができるため,これを実装する.速度に関しては,赤方偏移係数に光速の値をかけることで求めることができる.
    \inputminted[linenos, firstline=29, lastline=31, frame=lines, fontsize=\small]{matlab}{code/Exp3_6_Matlab.m}

    % svgに変更しよう
    \begin{figure}[H]
        \centering
        \includegraphics[width=0.5\textwidth]{figure/stellar1.pdf}
        \caption{恒星 HD94028 のスペクトルと水素アルファ線}
        \label{fig:exp3_6_spectra}
    \end{figure}

    ここから,各星の水素アルファ線を求める.行列の各行に対して最小値とそのインデックスを計算する.これは \mintinline{matlab}|min(spectra(:,:))| で求めることができる.また,各星の Ha の波長を \mintinline{matlab}|lambda(idx)|で求め,その値を用いて赤方偏移係数と速度を求める.

    \inputminted[linenos, firstline=11, lastline=15, frame=lines, fontsize=\small]{matlab}{code/Exp3_6_2_Matlab.m}

    この各星で求めた値を1つの図にまとめるプログラムで出力する.全ての星のスペクトルを一つのグラフに目止めて描画し,青方偏移か赤方偏移を視覚的に区別し,どの線がどの星に対応するかを明確にする.各星について速度が 0 以下の場合は青方偏移,0 より大きい場合は赤方偏移と判断するため,if文を用いて条件分岐を行う.また,その条件分岐を各星に対して行うために for 文を用いる.
    \inputminted[linenos, firstline=16, lastline=27, frame=lines, fontsize=\small]{matlab}{code/Exp3_6_2_Matlab.m}

    % svgに変更しよう
    \begin{figure}[H]
        \centering
        \includegraphics[width=0.5\textwidth]{figure/stellar2.pdf}
        \caption{各星のスペクトルと水素アルファ線}
        \label{fig:exp3_6_spectra2}
    \end{figure}

\subsection{第七回の方法}
    まず,刺激呈示環境を構築する.まず,背景色を白色とするために \mintinline{matlab}|bgcolor = 255| と設定する.刺激画像の大きさは 640 $\times$ 480 とするので 高さは \mintinline{matlab}|h = 640| とし,幅は \mintinline{matlab}|w = 480| とする.

    \begin{program}
    \inputminted[linenos, firstline=5, lastline=12, frame=lines, fontsize=\small]{matlab}{code/Exp3_7_Matlab.m}
    \end{program}

    画面左右端と画像端との距離は 220 ピクセルとするため, \mintinline{matlab}|margin = 220| とする.固視点は 画像中央に幅 40,縦 4 及び 幅 4 縦 40 の黒色の十字とする.黒色は 0 で指定することができ,固視点の十字は長方形を二つで組み合わせることで表現できる.ここで長方形は \mintinline{matlab}|[left top right bottom]| でベクトルで定義できる.
    \begin{program}[H]
        \inputminted[linenos, firstline=32, lastline=36, frame=lines, fontsize=\small]{matlab}{code/Exp3_7_Matlab.m}      
        \caption{固視点の表示}
        \label{lst:exp3_7_eye}
    \end{program}

    固視点呈示後,準備された左右の顔画像を表示する.それはコード\ref{lst:exp3_7_window}で表示することができる.
    \begin{program}[H]
        \caption{刺激の呈示}
        \inputminted[linenos, 
        firstline=95,
        lastline=99,
        frame=lines,
        fontsize=\small]{matlab}{code/Exp3_7_Matlab.m}
        \label{lst:exp3_7_window}
    \end{program}

    刺激を呈示した後の回答を格納するための関数は \mintinline{matlab}|KbWait([], 3)| を用いる.KbWait の返り値の一つである keyCode1 で押されたキーが確認できるが 1 $\times$ 256 の論理配列であり,各キーに対応した要素を真である.keycode1 で真になった要素を探索するため関数 Find を使用して行った.そのコードが\ref{lst:exp3_7_get}である.

    \begin{program}[H]
        \caption{反応時間の取得}
        \inputminted[linenos,
            firstline=102,
            lastline=107,
            frame=lines,
            fontsize = \small
        ]{matlab}{code/Exp3_7_Matlab.m}
        \label{lst:exp3_7_get}
    \end{program}

    これで,全ての試行が完了した後は,results配列にデータを格納するコードをコード\ref{lst:exp3_7_result}に示す.

    \begin{program}[H]
        \caption{データの保存}
        \inputminted[linenos,
        firstline=106,
        lastline=111,
        frame=lines,
        fontsize = \small]{matlab}{code/Exp3_7_Matlab.m}
        \label{lst:exp3_7_result}
    \end{program}

    最後に,results配列の値を csv ファイルに書き出した保存する.ここでは writematrix 関数を使用して \mintinline{matlab}| writematrix(results, filename)| と記述することで書き出すことができる.ここでの filename は,exp3\_7\_05.csv のことであり,拡張子に csv がついているため csv ファイルで保存される.
        \begin{program}
        \caption{データの書き出し}
        \inputminted[linenos,
        firstline=129,
        lastline=132,
        frame=lines,
        fontsize = \small]{matlab}{code/Exp3_7_Matlab.m}
        \label{lst:exp3_7_write}
    \end{program}


    \subsection{第七回の結果}
    第七回では眼球運動計測実験で使用するプログラムを作成した.このプログラムを実行した後に書き出される csv ファイルには 1 列目に 試行番号, 2 列目に 被験者が押したキーに対応するキーコード, 3 列目に反応時間が入力される.これらのデータを用いて眼球運動に関してのデータ分析をすることが期待できる.

    \subsection{第八回の方法}
    以下の 4 つの実験を行う.ただし,Eye Link II が故障しているため,第\ref{眼球} 目については実施をしない.
    \subsubsection{眼球運動計測実験 (1)}\label{眼球}
    時間計測度の高い眼球運動計測装置を用いて,顔の印象に関する判断を行う時に眼球運動の計測を行う.眼球運動計測装置として Eye Link II を用いる.まず,被験者に Eye Link II の愛カメラを装着し,キャリブレーション,バリデーションを行う.次に固視点を注視し,左右に現れる顔画像のどちらかが魅力的かを判断し,キー押しで反応してもらう.この試行を合計 20 試行行う.
    \subsubsection{眼球運動計測実験 (2)}
    メガネ型眼球運動計測装置である Tobii Glass 3 を用いて対象を自由に見る時の眼球運動を解析する.実験の手続きとしてはアイトラッキング用メガネを装着し,呈示されるマーカを追視.視線方向のキャリブレーションを行う.ここで,対象を自由に観察した時の Gaze Plot と Heat Map を作成する.
    \subsubsection{脳波計測}
    タイプしたい文字を脳情報のみから指定するブレイン・マシン・インターフェースのシステムを体験する.脳波による指標は,刺激提示から約 300 ミリ秒ほど後に頂点をもって現れる陽性の脳電位である p 300 を用いる. MatLab から Simulink のブロックセットを起動する.ドライ電極を頭部に装着する.電極位置は電波キャップに従って決める.本実験では視覚的刺激を提示する.
    \subsubsection{筋電計測}
    最大随意収縮及びダンベルを持っている時の上腕二頭筋の筋活動を計測する.まずは,LabChartを立ち上げる.今回は上腕二頭筋から筋活動を計測する.設置電極は肩峰とする.電極間の距離は約 2cm とする.サンプリング周波数は 1000Hz とする.

    最大随意収縮の計測については,安静状態から腕を 90 度曲げた状態で机等の動かないものに対して等尺性収縮で最大筋力を発揮する.ダンベルについては,1kg, 3kg, 5kg を用いて安静状態から選んだダンベルを 90 度曲げた状態で保持中の筋活動を計測する.

    \subsection{第九回の方法}
    結果を格納するための行列を作成する.行数はサンプリング周波数に対応する 500 行と列数は時間データ列と試行数となる.時間軸のデータについてはキー押しの 1 秒前からのデータを用いる.キーを押した時を 0msec として -998 までの 2msec おきの配列を作成する.これらを実装したコードがコード\ref{lst:exp3_9_vec}である.
    
    \begin{program}
        \caption{結果格納用の準備}
        \inputminted[linenos,
        firstline=1,
        lastline=11,
        frame=lines,
        fontsize = \small]{matlab}{code/Exp3_9_Matlab.m}
        \label{lst:exp3_9_vec}
    \end{program}

    ここから各試行ごとにデータ処理を行っていく.キー押しの瞬間を基準とした相対的な時間軸を用いるため,キーを押した時刻を 0 の基準として,元々格納している時間である絶対時間を用いて相対時間を求める.それが以下のコード\ref{lst:exp3_9_expos}である.
    \begin{program}[H]
        \caption{注視データの整形}
        \inputminted[linenos,
        firstline=18,
        lastline=27,
        frame=lines,
        fontsize = \small]{matlab}{code/Exp3_9_Matlab.m}
        \label{lst:exp3_9_expos}
    \end{program}

    ここで,欠損値についても考えていく.欠損値とは,眼球運動の測定データにおいて,特定の時点でデータが取得されなかった場合や解析するにあたって無効なデータの場合は欠損として扱われる. EyeLink の測定では,まばたきやセンサの誤作動によってデータが欠けることがあるため, MatLab 上では欠損値として扱う.欠損データは -1 で初期化する.また,有効な行のみを対象に画面状態や X 座標を代入する.有効な行の識別は \mintinline{matlab}|ismember(expos(:, 1), eye_data(:, 1) - eye_data(end, 1))| でき, expos の各時間点で実際の眼球データが存在するかを確認する.
    その結果から有効な行のインデックスを取得する.expos の各列は以下の形式で入力を行う.
    \begin{enumerate}[label=\arabic*列目]
        \item 時間
        \item 測定する目の情報
        \item 両目の状態
        \item 左目注視位置 X座標
        \item 左目注視位置 Y座標
        \item 左目瞳孔径
    \end{enumerate}
    この列に対応した情報を入れていくが有効な行のみを対象とするため, ismember で取得したインデックスを用いて行う.これらのコードをコード\ref{lst:exp3_9_missing}に示す.

    \begin{program}
        \caption{欠損値の処理}
        \inputminted[linenos,
        firstline=29,
        lastline=48,
        frame=lines,
        fontsize = \small]{matlab}{code/Exp3_9_Matlab.m}
        \label{lst:exp3_9_missing}
    \end{program}

    これから,注視位置の判定を行う.画像提示時のうち,x 座標を取得し,これらの平均値を計算する.この平均値の四捨五入したものがその試行における画面の中心 x 座標とする.この中心よりも左であるならば左側の画像を選択したと判断し,右であれば右側の画像を選択したと判断する.ここで,左側を見ていた場合を 100, 右側を見ていたなら 102 とする.画像提示時は expos の 3 列目に格納されており,その時の値は 2 であれば刺激画像を提示してる時である.また,左目の注視位置は, 5 列目に格納されており,中心よりも左であるか右であるかの論理積を取ることでそれぞれ求めることができる.これらのコードをコード\ref{lst:exp3_9_select}に示す.
    \begin{program}[H]
        \caption{注視位置の判定}
        \inputminted[linenos,
        firstline=50,
        lastline=58,
        frame=lines,
        fontsize = \small]{matlab}{code/Exp3_9_Matlab.m}
        \label{lst:exp3_9_select}
    \end{program}

    注視方向と選択の位置判定はキー押しの選択と視線が一致していたら 1, 一致していなければ 0 とする.これを行うために expos の 3 列目の値が 2 の時に選択位置と注視位置の判定を行う.また,選択位置は 100, 102 のどちらかであり,注視位置は -1, 1 のどちらかであるため,それらの論理積を取ることで一致しているかを判定する. どちらの顔画像が魅力的かを判断したデータは data という行列の 2 行目に格納されているため,そのデータとキーコードが一致しているかを確認する.また,キー押し直線の一秒間に限定して,被験者が最終的に魅力的だと判断した画像を見ていたかどうかをのデータを格納する.
    これらを実装したコードをコード\ref{lst:exp3_9_match}に示す.
    \begin{program}[H]
        \caption{注視位置と選択位置の一致}
        \inputminted[linenos,
        firstline=60,
        lastline=76,
        frame=lines,
        fontsize = \small]{matlab}{code/Exp3_9_Matlab.m}
        \label{lst:exp3_9_match}
    \end{program}
    コード\ref{lst:exp3_9_match_rate}では,全試行のデータが格納されている \mintinline{matlab}|data_matrix| をもとに,キー押し前の各時点において,平均してどのくらいの割合で被験者が最終的に選択した魅力的な画像をを見ていたのかを求める.500 行 1 列のゼロベクトルを用意する.その後,そのベクトルに抽出されたデータの平均値を計算する.0 と 1 のデータセットの平均値は,1 の割合を表す.したがって,\mintinline{matlab}|mean_valid_data| には, j の場合において有効な試行データの中で魅力的な画像を見ていた試行の割合が格納される.
    \begin{program}
        \caption{注視位置と選択位置の一致率}
        \inputminted[linenos,
        firstline=79,
        lastline=91,
        frame=lines,
        fontsize = \small]{matlab}{code/Exp3_9_Matlab.m}
        \label{lst:exp3_9_match_rate}
    \end{program}

    全試行分の統計として,各時間ごとに視線が魅力的な画像と一致していた割合を計算する.結果を時間軸に沿ってグラフ描画する.これをコード\ref{lst:exp3_9_plot}に示す.
    \begin{program}[H]
        \caption{注視位置と選択位置の一致率のグラフ化}
        \inputminted[linenos,
        firstline=92,
        lastline=96,
        frame=lines,
        fontsize = \small]{matlab}{code/Exp3_9_Matlab.m}
        \label{lst:exp3_9_plot}
    \end{program}

    \subsection{第十回の方法}
    まず,読み込んだテキストファイルから筋電データを読み込む.次に,筋電データを plot 関数を用いて読み込んだ各条件のデータを描画し,視覚的に確認するコードをコード\ref{lst:3_10_plot}に示す.

    \begin{program}[H]
        \caption{筋電データの読み込みと描画}
        \inputminted[linenos,
        firstline=1,
        lastline=25,
        frame=lines,
        fontsize=\small]{matlab}{code/Exp3_10_Matlab.m}
        \label{lst:3_10_plot}
    \end{program}
    次に,最大随意収縮 (MVC) の解析を行う.MVC のデータは,dataBlocks{1} に格納されているため,このデータを用いる.コード\ref{lst:3_10_plot} で,表示したデータから筋活動が安定している安静時区間を目視で確認し,その区間の平均値をベースライン値として算出する.そこから,MVC の値を引くことで基線のずれを補正する.また,全波整流を行うためにベースライン補正後の MVC データを abs 関数を用いて絶対値化する.最後に,整流処理後のデータから max 関数を用いて最大値を取得する.
    \begin{program}
        \caption{最大随意収縮 (MVC) の解析}
        \inputminted[linenos,
        firstline=27,
        lastline=48,
        frame=lines,
        fontsize = \small]{matlab}{code/Exp3_10_Matlab.m}
        \label{lst:3_10_mvc}
    \end{program}

    これから,MVC 以外の各負荷条件のデータに対して,処理を行なっていく.最初に,条件データの準備を行う.dataBlocks{2} から dataBlocks{5} までのデータをループを用いて抽出していき,抽出したデータの安静時区間をコード\ref{lst:3_10_plot} で出力されたグラフから出力されているため,目視で確認する.その区間の筋電位の平均値をベースラインとして算出後,データ全体から減算し,整流する.これらの実装をコード\ref{lst:3_10_load}に示す.
    \begin{program}[H]
        \caption{各負荷条件のデータの準備}
        \inputminted[linenos,
        firstline=53,
        lastline=69,
        frame=lines,
        fontsize = \small]{matlab}{code/Exp3_10_Matlab.m}
        \label{lst:3_10_load}
    \end{program}

    次に,ローパスフィルタ処理を設定する.サンプリング周波数を 1000Hz, カットオフ周波数 5Hz の2次バターワースローパスフィルタを butter 関数で設計する.この設定したファイルを filtfit 関数を用いて整流後のデータに適用し,平滑化を施す.filtfit 関数は位相の歪みを防ぐために順方向と逆方向からフィルタをかける.このフィルタ処理後のデータから,筋が安定して活動していると見なせる 1000ms 区間の平均値を求める.この筋が安定していると見なせる区間はコード\ref{lst:3_10_plot} で出力された図で確認できる.ここから特定した安定活動区間が適切であるかを確認するため,フィルタ処理後のデータと選択された区間の開始・
    終了位置をプロットする.ここで,分散が最も小さくなる区間をとする.これらの内容をコード\ref{lst:3_10_ana} に記す.

    \begin{program}
        \caption{各負荷条件のデータの解析}
        \inputminted[linenos,
        firstline=71,
        lastline=90,
        frame=lines,
        fontsize = \small]{matlab}{code/Exp3_10_Matlab.m}
        \label{lst:3_10_ana}        
    \end{program}

    適切な区間が正しいかをグラフに出力して確かめ,正規化筋電を求め,結果を格納するベクトルに結果を入れる.これらの内容をコード\ref{lst:3_10_norm}に示す.

    \begin{program}
        \caption{各負荷条件の正規化}
        \inputminted[linenos,
        firstline=71,
        lastline=90,
        frame=lines,
        fontsize = \small]{matlab}{code/Exp3_10_Matlab.m}
        \label{lst:3_10_norm}
    \end{program}

    各条件の解析終了後,結果をグラフで表示する.

    \begin{program}
        \caption{各負荷条件の正規化}
        \inputminted[linenos,
        firstline=142,
        lastline=157,
        frame=lines,
        fontsize = \small]{matlab}{code/Exp3_10_Matlab.m}
        \label{lst:3_10_norm}        
    \end{program}

    \section{各回の結果}

    \subsection{第九回の結果}
    これらの結果より,図\ref{fig:exp3_9_plot} を得ることができる.この図から読み取れることは,-1000ms 付近では割合は比較的低く,どちらかの画像を注視しているかが明確でないか選択していない画像を注視している可能性がある.時間が 0ms に近づくにつれ割合が高くなっており魅力的な画像を注視していることがわかる.また,-500ms, -300ms あたりでは割合が高くなっていることがわかる.最後の 0ms 付近では非常に高い割合に達しており,被験者の視線はほぼ魅力的だと判断した画像に向けられていることを意味していると考えられる.まとめとして,被験者が顔画像の魅力を判断し,キーを教えて選択するまでの最後の 1 秒間で意思決定が固まるにつれ,最終的に選択する画像に対する注視が高まっていく過程を数値化している.
    \begin{figure}[H]
        \centering
        \includegraphics[width=0.5\textwidth]{figure/g0310/lab3_2.png}
        \caption{注視位置と選択位置の一致率}
        \label{fig:exp3_9_plot}
    \end{figure}

    \subsection{第十回の結果}
\end{document}