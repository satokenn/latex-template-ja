\documentclass[divpdfmx]{jlreq}
% \usepackage{geometry}
\usepackage{ascmac}
\usepackage{graphicx}
\usepackage{amsmath}
\usepackage{amssymb}
\usepackage{listings}
\usepackage{float}
\usepackage{fix-cm}
\usepackage{listings}

\usepackage{hyperref,graphicx}
% for hyperref
% \usepackage{pxjahyper}
\hypersetup{
	colorlinks=false,% リンクに色をつけない設定
	bookmarks=true,% 以下ブックマークに関する設定
	bookmarksnumbered=true,
	pdfborder={0 0 0},
	bookmarkstype=toc
}

\lstset{
  basicstyle={\ttfamily},
  identifierstyle={\small},
  commentstyle={\smallitshape},
  keywordstyle={\small\bfseries},
  ndkeywordstyle={\small},
  stringstyle={\small\ttfamily},
  frame={tb},
  breaklines=true,
  columns=[l]{fullflexible},
  numbers=left,
  % xrightmargin=0zw,
  % xleftmargin=3zw,
  numberstyle={\scriptsize},
  stepnumber=1,
  numbersep=1,
  lineskip=1ex
}

\title{情報学群実験第3}
\author{情報学群 \\ 1270328 佐藤謙成}
\date{\today}
\begin{document}
\maketitle
\section{目的}
本レポートでは,IPアドレスをDNSサーバと接続することで自身が持っているドメインとIPアドレスを接続する.ドメインとIPアドレスを接続することで32ビットの数字から文字列に変換することができるため,
人から見ても理解しやすい名前として利用することができる.また,サーバの移転等でIPアドレスが変更した際も紐付けなおすことでユーザには変更を意識させることなくサイトにアクセスすることができる.
また,ドメインとIPアドレスを接続しないデメリットとしてサーバのIPアドレスを直接入力する必要があり,この作業は非常に煩雑かつタイプミスなども発生しやすいため,サイトへのアクセスは困難となる.
\section{内容}
今回行った作業は,他ドメイン向けにDNSで自サーバ名の名前提供を行うサービスを提供する.
\section{要素技術}
ネットワーク層プロトコルではIPアドレスという32ビットの整数で表される.しかし,ホストの識別は
\section{作業記録}
まず,パッケージサーバに最新情報を紹介してBIND9をインストールする.
% \begin{lstlisting}
%     sudo apt update
%     sudo apt install bind9
% \end{lstlisting}
を実行する.その後,設定ファイルをvimで編集する.これから編集するファイルは以下の5種類である.
\begin{enumerate}
    \item /etc/bind/named.conf.default-zones
    \item /etc/bind/named.conf.options
    \item /etc/bind/named.conf.local
    \item /etc/bind/named/270328f.zone
\end{enumerate}
それぞれ,以下の絶対パスであるのでそれらをcpコマンドでコピーし,別名で待避させたのちにvimで開き編集する.最初に named.conf.default-zonesで自分が管理する
ドメインをzoneで指定する.
% \begin{lstlisting}[label = zoneで指定する]
% zone "270328f.exp.kut.jp" {
%     type primary;
%     file "/etc/bind/270328f.zone";
% }
% \end{lstlisting}
つぎに,BINDの全体設定が終わったら,自分のドメインの情報を270328f.zoneファイルに記述する.
% \begin{lstlisting}[label = ゾーンファイルへの記述]
%     $TTL 100
%     @ IN SOA ns.270999x.exp.kut.jp. postmaster.270999x.exp.kut.jp. (
%     2025040901
%     100
%     100
%     100
%     100 )
%     IN NS ns.270999x.exp.kut.jp.
%     ns IN A サーバの Elastic IP
%     www IN CNAME ns
% \end{lstlisting}
これを記述した後に,BINDを再起動する.
% \begin{lstlisting}
% sudo systemctl reload bind9
% \end{lstlisting}
% \begin{lstlisting}{DNSサーバの再起動と動作確認}
% sudo systemctl restart bihd9
% sudo ps auxqq | grep named
% \end{lstlisting}
ここで,53番ポートを開放してPCからドメインを検索することでサイトに入れるかを確認する.
ここまででドメインの名前解決をすることができたため,次はnginxの設定を修正する.
% \begin{lstlisting}
%     server_name www.270328f.exp.kut.jp;
% \end{lstlisting}
この文を追加することでnginxが受け取るHTTPリクエストのうち,Hostヘッダーがwww.270328f.exp.kut.jpになっているアクセスに対して対応するという意味であり,
これを実装することでドメインとnginxの設定を紐付けることができた.

\subsection{認証および暗号化}
これからHTTP通信の暗号化をSSL/TLSを用いて行う.
暗号化されたHTTPS通信の実現は,以下の手順で行うことができる.
\begin{enumerate}
    \item 仮の秘密鍵・証明書(暗号化のみ実現)
    \item 秘密鍵・自己署名証明書
    \item 秘密鍵・PKI基盤での証明書
\end{enumerate}
まず,HTTP認証の証明書の設定としてhtpasswdを用いてユーザ・パスワードの設定をする.
%\begin{lstlisting}
%   sudo htpasswd -c /etc/apache2/.htpasswd testuser
%\end{lstlisting}
このコマンドを打つことで.htpasswdが無い際に作成し,"testuser"というユーザを作成して,そのパスワードを入力する.また,この認証を実際のサイトで適用するために
/var/www/htmlの下にbasicというディレクトリを作成し,認証を行う設定をするには/etc/apache2/sites-available/000-default.confを管理者権限でvimを用いて開く.
以下のコードが実際の設定である.
\begin{lstlisting}
    <Directory /var/www/html/basic>
        AuthType Basic
        AuthUserFile /etc/apache2/.htpasswd
        AuthName "basic auth"
        Satisfy any
        Order deny,allow
        Deny from all
        Require valid-user
    </Directory>
\end{lstlisting}
次は,Let's Encryptによる証明書を作成する.Let's Encryptは無料でDV証明書を発行してくれる認証局である.
以下の手順を用いて署名を行う.
\begin{enumerate}
    \item サーバ側のエージェントソフトウェアが鍵ペアを作成
    \item www.270328f.exp.kut.jpを管理していることをLet's Encryptの認証局に証明するため,どちらかを実行する
    \begin{itemize}
        \item www.270328f.exp.kut.jpのDNSレコードを認証局に提示
        \item http://www.270328f.exp.kut.jpの下にHTMLのコンテンツを配置
    \end{itemize}
    \item 認証局はエージェントソフトウェアにノンスを提供
    \item 秘密鍵でノンスに署名して,それを認証局へ送付
    \item 認証局は署名を検証
    \item 認証局はチャレンジに対する処理を確認
    \item 証明書署名要求内部の秘密鍵と証明書署名要求に付与されている署名を検証して,正当であれば,認証局の署名を付与したwww.270328f.exp.kut.jpの証明書をエージェントソフトウェアに返信
\end{enumerate}
これらの手順を今から実際に行っていく.
\begin{lstlisting}
    $ sudo apt update
    $ sudo apt install python3 python3-venv libaugeas0
    $ sudo python3 -m venv /opt/certbot/
    $ sudo /opt/certbot/bin/pip install --upgrade pip
    $ sudo /opt/certbot/bin/pip install certbot certbot-nginx
    $ sudo ln -s /opt/certbot/bin/certbot /usr/bin/certbot
    j$ sudo certbot --nginx
\end{lstlisting}
\section{考察}
\end{document}