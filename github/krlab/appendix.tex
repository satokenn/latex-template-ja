\documentclass{jsarticle}
\usepackage{amsmath}
\usepackage{graphicx}
\begin{document}

\section*{補足説明}

前側被写界深度 $a - a_n$ と後側被写界深度 $a_f - a$ は、つぎのように被写体距離 $a$、許容錯乱円 $e$、レンズ絞り値 $F$、レンズ焦点距離 $f$ により計算できる。

撮影レンズを薄肉レンズで近似する。実際の撮影レンズは前側主点と後側主点をもつ厚肉レンズだが、前側主点と後側主点間の距離は被写体距離と比較して小さいため無視する。薄肉レンズでは、式(2.5)のガウスのレンズ公式から、以下の関係がある。

\begin{equation}
\frac{1}{a} + \frac{1}{b} = \frac{1}{f}, \quad
\frac{1}{a_n} + \frac{1}{b_n} = \frac{1}{f}, \quad
\frac{1}{a_f} + \frac{1}{b_f} = \frac{1}{f}
\tag{2.14}
\end{equation}

さらに、図2.24に示すように、許容錯乱円$e$とレンズ絞り値$F$では、式(2.15)の関係がある。

% \begin{figure}[h]
% \centering
% \includegraphics[width=0.8\linewidth]{triangle_diagram.png} % 図を適切に挿入
% \caption{図2.24 — 比はすべてレンズ絞り値 $F$}
% \end{figure}

\begin{align}
\frac{f}{D} = F &\Rightarrow D = \frac{f}{F} \notag \\
\frac{(b_n - b)}{e} = F &\Rightarrow b_n = b + eF = \frac{af}{a - f} + eF \notag \\
\frac{(b - b_f)}{e} = F &\Rightarrow b_f = b - eF = \frac{af}{a - f} - eF
\tag{2.15}
\end{align}

式(2.15)を式(2.14)に代入して整理すると、

\begin{align}
a - a_n &= \frac{e(a - f)^2 F}{f^2 + e(a - f)F} = \frac{ea^2 F}{f^2 + eaf} \tag{2.16} \\
a_f - a &= \frac{e(a - f)^2 F}{f^2 - e(a - f)F} = \frac{ea^2 F}{f^2 - eaf} \tag{2.17}
\end{align}

ただし、被写体距離 $a$ がレンズ焦点距離 $f$ と比較して十分に大きいときには、最右辺のように $(a - f) \approx a$ と近似している。

たとえば、フルサイズ撮像素子を搭載したカメラに焦点距離 $f = 35$mm のレンズを取り付け、レンズ絞り値を $F = 2.8$、被写体までの距離 $a = 2$m、許容錯乱円を $e = 43.3 \times \frac{1}{1500} = 0.029$mm とすると、前側被写界深度 $a - a_n = 233$mm、後側被写界深度 $a_f - a = 304$mm で、前後合計 $a_f - a_n = 537$mm、1.8m から 2.3m までピントが合う。

1/3インチ撮像素子を搭載したカメラに焦点距離 $f = 4.8$mm のレンズを取り付けた場合と同じ撮影画角になるように、撮影レンズの焦点距離を選定することができる。たとえば、1/3インチ撮像素子を搭載したカメラに焦点距離 $f = 4.8$mm のレンズを取り付けると、35mmフィルム換算焦点距離が $4.8 \times 43.3 / 6.0 = 34.6$mm となる。フルサイズ撮像素子を搭載したカメラに焦点距離 $f = 35$mm のレンズを取り付けた場合と撮影画角がほとんど変わらない。

同じレンズ絞り値に設定したときの被写界深度は大きく異なる。1/3インチ撮像素子を搭載したカメラでは、許容錯乱円を $e = 60.0 \times \frac{1}{1500} = 0.004$mm とすると、前側被写界深度 $a - a_n = 986$mm、後側被写界深度 $a_f - a = 7000$mm となる。つまり、約1mから7.2mまでピントが合う。

撮影画角は同じでも被写界深度の大きさが異なるのは、許容錯乱円の大きさが1乗で、焦点距離は2乗で被写界深度に影響するためである。

1/3インチ撮像素子を搭載したカメラに焦点距離 $f = 4.8$mm のレンズを取り付けてピント位置を $a = 2$m に設定すると、1mから8m以上の無限遠までピントが合う。このようにきわめて広い範囲にピントを合わせることを、パンフォーカスという。

\end{document}
