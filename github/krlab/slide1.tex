% コピペスタート
\documentclass[8pt, jfont=ipaexm]{beamer} % IPAex明朝
\usepackage{hyperref}
% 一般的によく使用されるパッケージ
\usepackage[utf8]{inputenc}
\usepackage{graphicx}
\usepackage{amsmath}
\usepackage{amsfonts}
\usepackage{amssymb}
\usepackage{tikz}
\usepackage{pgfplots}
%\usepackage{xeCJK}
%\usetheme{Copenhagen}
\pgfplotsset{compat=1.17}
\usepackage{listings}
% \usepackage{slashbox}

% スタイル設定
\usepackage{ifthen}
\usepackage[varg]{txfonts}
\usepackage{ragged2e}
\usepackage{svg}
\usepackage{xcolor}
\usepackage{url}
\usepackage{bm}
\usepackage{pgfplots}
\pgfplotsset{compat=1.18}

\usetikzlibrary{graphs}
\usetikzlibrary {arrows.meta}
\usetikzlibrary {bending}
\usetikzlibrary{arrows,shapes,automata,petri,positioning,calc}


% 和文用パッケージ(luatex用)
\usepackage{luatexja}
\usepackage{luatexja-fontspec}
\usepackage{lmodern}
\usepackage[T1]{fontenc} % 必要に応じてフォントエンコーディングを指定


%カラーテーマの選択(省略可)
\usecolortheme{orchid}
%フォントテーマの選択(省略可)
\usefonttheme{professionalfonts}
%フレーム内のテーマの選択(省略可)
\useinnertheme{circles}
%フレーム外側のテーマの選択(省略可)
\useoutertheme{infolines}
%しおりの文字化け解消
% \AtBeginShipoutFirst{\special{pdf:tounicode EUC-UCS2}}

% \AtBeginShipoutFirst{\special{pdf:tounicode 90ms-RKSJ-UCS2}}
%ナビゲーションバー非表示
%\setbeamertemplate{navigation symbols}{}

% タイトル色
\setbeamercolor{title}{fg=structure, bg=}

% フレームタイトル色
\setbeamercolor{frametitle}{fg=structure, bg=}

% caption に番号追加
\setbeamertemplate{caption}[numbered]
% caption 日本語
\renewcommand{\figurename}{図}
\renewcommand{\tablename}{表}

\usepackage[export]{adjustbox} % loads also graphicx


\usetheme[progressbar=frametitle, block=fill, numbering=fraction,]{metropolis}
% \usetheme{default}
            
% ブロックのスタイルをカスタマイズ
\setbeamertemplate{blocks}[rounded]
\setbeamercolor{block title}{bg=gray!30,fg=black} % ブロックのタイトルの背景とフォントの色
\setbeamercolor{block body}{bg=gray!10,fg=black} % ブロック本体の背景とフォントの色

\setbeamercolor{block title example}{bg=orange!30,fg=black} % 例のブロックのタイトルの背景とフォントの色
\setbeamercolor{block body example}{bg=orange!10,fg=black} % 例のブロック本体の背景とフォントの色

\setbeamercolor{block title alerted}{bg=red!30,fg=black} % アラートのブロックのタイトルの背景とフォントの色
\setbeamercolor{block body alerted}{bg=red!10,fg=black} % アラートのブロック本体の背景とフォントの色

\tikzset{set label/.style={fill=white,circle,inner sep=2}}

\def\radius{2}
\def\ratio{0.6}

\def\centerA{180:\ratio*\radius}
\def\circleA{(\centerA) circle [radius=\radius]}





%追加
\setbeamertemplate{footline}{%
  \hfill%
  \usebeamercolor[fg]{page number in head/foot}%
  \usebeamerfont{page number in head/foot}%
  \insertframenumber\,/\,\inserttotalframenumber\kern1em\vskip2pt%
}

%ソースコードに関する設定
\lstset{
  basicstyle={\ttfamily},
  identifierstyle={\small},
  commentstyle={\smallitshape},
  keywordstyle={\small\bfseries},
  ndkeywordstyle={\small},
  stringstyle={\small\ttfamily},
  frame={tb},
  breaklines=true,
  columns=[l]{fullflexible},
  numbers=left,
  % xrightmargin=0zw,
  % xleftmargin=3zw,
  numberstyle={\scriptsize},
  stepnumber=1,
  numbersep=1,
  lineskip=1ex
}


\tikzset{
    place/.style={
        circle,
        thick,
        draw=black,
        fill=gray!50,
        minimum size=6mm,
    },
        state/.style={
        circle,
        thick,
        draw=blue!75,
        fill=blue!20,
        minimum size=6mm,
    },
}



\title{離散数学}
\author{佐藤謙成}
\date{\today}
\begin{document}

\maketitle

\begin{frame}{目次}
    \tableofcontents
\end{frame}

\section{写像}

\begin{frame}{写像}
    \begin{block}{}
        2つの集合 $A$ から $B$ への関係のうち, 集合 $A$ の各要素に, それぞれ $B$ の
        要素がただ一つだけ対応している関係を $A$ から $B$ への写像という. この時, 集合 $A$ を定義域と呼び, 集合 $B$ を値域, 像と呼ぶ. $f$ が $A$ から $B$ への写像ならば,
        \begin{equation*}
            f : A \to B
        \end{equation*}
        と書く. ここで, $a$ の対応先が $b$ のとき $b = f(a)$ と書く. また, 関数 $f$ の要素でもあるので順序対 $(a, b)$ と書くことができる.
    \end{block}
\end{frame}

\begin{frame}{恒等写像}
    \begin{block}{}
        $A$ を任意の集合とし, $A$ の各要素 $a$ に $a$ を対応させれば, これも $A$ から $A$ への写像となる. この写像を恒等写像という.
    \end{block}
\end{frame}

\begin{frame}{単射}
    \begin{block}{}
        $A$ から $B$ への写像を $f$ とする. 定義域 $A$ の異なる要素が異なる像を持つならば, 写像 $f$ は単射という. $a_1, a_2 \in A$ について
        \begin{equation*}
            a_1 \neq a_2 \text{ならば} f(a_1) \neq f(a_2)
        \end{equation*}
    \end{block}
\end{frame}

\begin{frame}{全射}
    \begin{block}{}
        $A$ から $B$ への写像を $f$ とする.
        値域 $B$ の各要素が $A$ のある要素の像となっている時, 写像 $f$ は全射という.
        \begin{equation}
            \forall b \in B, \exists a \in A, b = f(a)
        \end{equation}
    \end{block}
\end{frame}

\begin{frame}{全単射}
    \begin{block}{}
    $f$ が単射かつ全射のとき, 全単射という. 1対1対応という.
    \end{block}
\end{frame}


\begin{frame}{逆写像}
    \begin{block}{}
        写像 $f : A \to B$ の逆関係 $f^{-1} : B \to A$ が写像となるための必要十分条件は, $f$ が $A$ から $B$ への全単射であることである. またそのときの $f^{-1}$ は $B$ から $A$ への全単射となる. これを $f$ の逆写像という.
    \end{block}
\end{frame}

\begin{frame}{写像の合成}
    \begin{block}{}
        $A, B, C$ を3つの集合とし, $A$ から $B$ への写像を $f$, $B$ から $C$ への写像を $g$ とする. ここで, $A$ の各元 $a$ に対して, $f$ による像として $B$ の元 $f(a)$ が定まり, さらに $f(a)$ の $g$ による像として $C$ の元, $g(f(a))$ が定まるので, すなわち $a$ に $g(f(a))$ を対応させる $A$ から $C$ への写像 $h: A \to C$ を考えることができる. この写像 $h$ を $f$ と $g$ との合成写像といい, $g \circ f$ または $(gf)$ で表す. 
    \end{block}
\end{frame}

\begin{frame}{合成写像の性質}
    \begin{block}{}
        $A, B, C$ を3つの集合とし, $A \to B$ への写像を $f$ , $B \ to C$ への写像を $g$ とすると
        \begin{enumerate}
            \item $f$ と $g$ が共に全射なら, $g \circ f$ は全射である.
            \item $f$ と $g$ が共に単射なら, $g \circ f$ は単射である.
            \item $f$ と $g$ が共に全単射なら, $g \circ f$ は全単射である.
        \end{enumerate}
    \end{block}
\end{frame}

\begin{frame}{合成写像の性質.2}
    \begin{block}{}
        複数の写像を合成する場合, 合成した結果は順番に依存しない, つまり任意の写像 $f, g, h$ について,
        \begin{equation*}
            (h \circ g) \circ f = h \circ (g \circ f)
        \end{equation*}
    \end{block}
    
\end{frame}

\begin{frame}{鳩の巣原理(部屋割り論法)}
    \begin{block}{}
        $U,V$が有限集合で$|U| > |V|$のとき,$U$から$V$への単射\footnote{始集合の任意の2つの要素が終集合の要素との関係で重複しないこと}は存在しない.\\
        言い換えると,$n+1$個以上のアイテムを$n$個の箱に入れる場合、少なくとも$1$つの箱には$2$個以上のアイテムが入る        
    \end{block}
\end{frame}

\begin{frame}{鳩の巣箱の原理.2}
    \begin{exampleblock}{問題例}
    $6$個の異なる正の整数をどのように選んでも,それらの中に,差が$5$で割り切れる$2$つの数が存在する.        
    \end{exampleblock}
    \begin{alertblock}{鳩の巣箱の原理の証明例}


    \mbox{}\\
        \( \mathbb{Z}^+ \) の部分集合で要素数が6となる集合 \( U \) を任意に取る。集合 \( V = \{0, 1, 2, 3, 4\} \) とし、\( U \) から \( V \) への写像 \( f \) を \( f(x) = x \mod 5 \) と定める。

\( |U| > |V| \) なので、鳩の巣原理より、\( V \) の要素 $y$で $|f^{-1}[\{y\}]| \geq 2$となる要素が存在する.そのような要素を $y_0$とする.

$f^{-1}[\{y_0\}]$から異なる要素 \( x_1 \) と \( x_2 \) を任意に取る。逆像の定義と \( f \) の定義より、\( f(x_1) = f(x_2) \)、即ち、\( x_1 \mod 5 = x_2 \mod 5 \) が成り立つ。modの定義より、\( x_1 - 5 \lfloor x_1 / 5 \rfloor = x_2 - 5 \lfloor x_2 / 5 \rfloor \) が成り立つ。

\( \mathbb{Z} \) の要素について、\( x_1 - x_2 = 5(\lfloor x_1 / 5 \rfloor - \lfloor x_2 / 5 \rfloor) \) が成り立つ。即ち、\( x_1 \) と \( x_2 \) の差は5で割り切れる。
ここで、\( U \) は任意の集合であったので、任意の異なる6個の正の整数からなる集合に対して、差が5で割り切れる2つの数が存在する。

    \end{alertblock}
\end{frame}
\end{document}
