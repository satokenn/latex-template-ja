% コピペスタート
\documentclass[xdvipdfmx, 8pt, t]{beamer}

% 一般的によく使用されるパッケージ
\usepackage[utf8]{inputenc}
\usepackage{graphicx}
\usepackage{amsmath}
\usepackage{amsfonts}
\usepackage{amssymb}
\usepackage{hyperref}
\usepackage{tikz}
\usepackage{pgfplots}
%\usepackage{xeCJK}
\usepackage{zxjatype}
\usepackage[ipaex]{zxjafont}
%\usetheme{Copenhagen}
\pgfplotsset{compat=1.17}
\usepackage{atbegshi}
\usepackage{listings}
% スタイル設定
\usepackage{ifthen}
\usepackage{otf}
\usepackage[varg]{txfonts}


%カラーテーマの選択(省略可)
\usecolortheme{orchid}
%フォントテーマの選択(省略可)
\usefonttheme{professionalfonts}
%フレーム内のテーマの選択(省略可)
\useinnertheme{circles}
%フレーム外側のテーマの選択(省略可)
\useoutertheme{infolines}
%しおりの文字化け解消
\usepackage{atbegshi}
\AtBeginShipoutFirst{\special{pdf:tounicode EUC-UCS2}}

\AtBeginShipoutFirst{\special{pdf:tounicode 90ms-RKSJ-UCS2}}
%ナビゲーションバー非表示
%\setbeamertemplate{navigation symbols}{}

% タイトル色
\setbeamercolor{title}{fg=structure, bg=}

% フレームタイトル色
\setbeamercolor{frametitle}{fg=structure, bg=}

% caption に番号追加
\setbeamertemplate{caption}[numbered]
% caption 日本語
\renewcommand{\figurename}{図}
\renewcommand{\tablename}{表}

\usepackage[export]{adjustbox} % loads also graphicx


\usetheme[progressbar=frametitle, block=fill, numbering=fraction,]{metropolis}
            
% ブロックのスタイルをカスタマイズ
\setbeamertemplate{blocks}[rounded]
\setbeamercolor{block title}{bg=gray!30,fg=black} % ブロックのタイトルの背景とフォントの色
\setbeamercolor{block body}{bg=gray!10,fg=black} % ブロック本体の背景とフォントの色

\setbeamercolor{block title example}{bg=orange!30,fg=black} % 例のブロックのタイトルの背景とフォントの色
\setbeamercolor{block body example}{bg=orange!10,fg=black} % 例のブロック本体の背景とフォントの色

\setbeamercolor{block title alerted}{bg=red!30,fg=black} % アラートのブロックのタイトルの背景とフォントの色
\setbeamercolor{block body alerted}{bg=red!10,fg=black} % アラートのブロック本体の背景とフォントの色




%追加
\setbeamertemplate{footline}{%
  \hfill%
  \usebeamercolor[fg]{page number in head/foot}%
  \usebeamerfont{page number in head/foot}%
  \insertframenumber\,/\,\inserttotalframenumber\kern1em\vskip2pt%
}

%ソースコードに関する設定
\lstset{
  basicstyle={\ttfamily},
  identifierstyle={\small},
  commentstyle={\smallitshape},
  keywordstyle={\small\bfseries},
  ndkeywordstyle={\small},
  stringstyle={\small\ttfamily},
  frame={tb},
  breaklines=true,
  columns=[l]{fullflexible},
  numbers=left,
  % xrightmargin=0zw,
  % xleftmargin=3zw,
  numberstyle={\scriptsize},
  stepnumber=1,
  numbersep=1,
  lineskip=1ex
}
% コピペフィニィッシュ
\title{C++入門}
\subtitle{第3回目}
\author{佐藤謙成}

\begin{document}
\begin{frame}
    \titlepage
\end{frame}

\begin{frame}<beamer>
\frametitle{目次}
    \tableofcontents[]
\end{frame}

\begin{frame}{注意}
    このスライドはtexのbeamerというドキュメントクラスを用いています.
\end{frame}

\section{配列と文字列}
\begin{frame}{1次元配列}
    \begin{block}{配列とは}
        全てが同じ型をもち,共通の名前によってアクセスされる変数のリストのこと.
        つまり,複数の変数をまとめたもの.
    \end{block}
    \begin{block}{配列の宣言}
        型 変数名[サイズ\footnote{配列の数.}]; 
        
        例として, int a[10];
    \end{block}
    \begin{alertblock}{}
        配列は0番目から始まるので、注意すること.
        サイズを10とした場合,0 \~ 9までの計10個が配列の添え字\footnote{添え字とは,各要素が何番目かを表すもの.0番目の要素の添え字は0.}となる.
    \end{alertblock}
\end{frame}

\begin{frame}{1次元配列の初期化}
1次元配列の初期化は以下のようにすることができる.ただし,競プロでは入力例から値を受け取るため使用することはほとんどない.
\begin{block}{1次元配列の初期化}
    型 配列名[サイズ] = \{値のリスト\};
    
    例として, int a[3] = \{1, 2, 3\};
    
\end{block}
\end{frame}

\begin{frame}{1次元配列と変数の違い.1}
    配列と変数の違いについて視覚的に分かりやすい図を用いて考えていく.
    \begin{figure}[h]
        \centering
        \includegraphics{}
        \caption{変数の図}
        \label{fig:enter-label}
    \end{figure}
    変数は一つ宣言すると一つの値しか代入することができない.
    \begin{figure}[h]
        \centering
        \includegraphics{}
        \caption{配列の図}
        \label{fig:enter-label}
    \end{figure}
    しかし、配列は一つ宣言しても自分が欲しい数の値だけ挿入することができる.(ただし,最初に宣言した個数しか代入することができない.)
\end{frame}
\begin{frame}{1次元配列と変数の違い.2}
    変数と配列の違いは「データがメモリ上に一列に並べられているかどうかである.」変数の場合は,メモリ上にバラバラに存在しているが配列は一列に並んでいる.そのため,配列を用いると各データへ効率よくアクセスすることができるというメリットがある.
\end{frame}

\begin{frame}{文字列}
    文字列型「String」は変数ではあるが,char\footnote{文字型}を1次元配列化したものとして考えることができる.
    \begin{figure}[h]
        \centering
        \includegraphics{}
        \caption{charの1次元配列}
        \label{fig:enter-label}
    \end{figure}
\end{frame}

\begin{frame}{文字列の使い方}
    文字列の使い方はC++のリファレンスを参照していただきたい.
    \url{https://cpprefjp.github.io/reference/string/basic_string.html}
    
    以下は,文字列を扱う際によく用いるものの代表例を取り上げている.
    \begin{table}[h]
        \centering
        \begin{tabular}{|l|l|}
        \hline
        名前 & 説明 \\
        \hline
        begin & 配列の要素の添え字を取得\\
        end & 最後の要素の添え字を取得\\
        \hline
        \end{tabular}
        \caption{イテレータ\footnote{ポインタと同様に扱うことができるもの.}}
        \label{tab:my_label}
    \end{table}

    \begin{table}[h]
        \centering
        \begin{tabular}{|l|l|}
        \hline
        名前 & 説明\\
        \hline
        size & 文字列の長さを取得\\
        \hline
        \end{tabular}
        \caption{領域}
        \label{tab:my_label}
    \end{table}
\end{frame}

\begin{frame}{多次元配列}
    1次元の配列だけでなく,2次元以上の配列を作成することができる.(2次元配列以上は使わなくてもプログラムを作成することは可能であるため,あまり使用しない.)
    \begin{block}{2次元配列の宣言}
        型 変数名[サイズ1][サイズ2];
        
        例として, int a[10][12];
    \end{block}
    以下は,2次元配列を視覚的に分かりやすくしたものである.
    \begin{figure}[h]
        \centering
        \includegraphics{}
        \caption{2次元配列の図}
        \label{fig:enter-label}
    \end{figure}
\end{frame}

\begin{frame}{2次元配列の初期化}
    2次元配列の初期化も1次元配列と同様に初期化することができる.
    \begin{block}{2次元配列の初期化}
        型 配列名[サイズ] = \{\{値のリスト1\}, \{値のリスト2\} $\cdots$\};
        
        例として, int a[2][2] = \{\{1, 2\}, \{1, 3\}\}

    \end{block}
    
    上の例を図で表した場合,以下のようになる.
    \begin{figure}[h]
        \centering
        \includegraphics{}
        \caption{2次元配列の初期化}
        \label{fig:enter-label}
    \end{figure}    

\end{frame}

\begin{frame}[fragile]{配列とfor文との組み合わせ.1}
    配列が一番有効活用できるときは,for文等の繰り返し処理を行う文法と組み合わせて使うときである.
    以下に,その例を挙げる.
    \begin{lstlisting}[caption=配列とfor文の組み合わせ,label=arr]
        #include <iostream>
        using namespace std;

        int main() {
            int n; // 配列aのサイズを何個にするか.
            cin >> n;
            int a[n];
            for(int i = 0; i < n; i++) { // 配列は0 ~ n-1個であるため
                cin >> a[i];
            }
            for(int i = 0; i < n; i++) { // 上のfor文と同様
                cout << a[i] << endl;
            }
        }
    \end{lstlisting}
\end{frame}

\begin{frame}{配列とfor文との組み合わせ.2}
    ソースコード\ref{arr}はサイズがnの配列にn回,for文を回すことにより,配列aの全要素に値を入れることができる.これは
    \begin{enumerate}
        \item n個の変数を一度に宣言し,
        \item n個の変数にfor文を回すだけで値を代入することができる
    \end{enumerate}
    ということができ,大量のデータを扱うことができるようになるのでぜひ使えるようにしていきたい.
\end{frame}
\subsection{可変長配列}
\begin{frame}{可変長配列"vector"}
    今までの配列は最初にサイズを指定した後にサイズを変更することができなかった固定長配列というものである.しかし,プログラムの実行途中に配列の要素の増減を行いたい場合も出てくる.そのような場合に,この可変長配列「vector」を用いる.
    \begin{block}{可変長配列とは}
        プログラミングで用いられる配列変数の一種であり,配列のサイズが固定されておらず,実行途中の必要に応じて要素を追加,削除することができる配列である.
    \end{block}
    \begin{alertblock}{可変長配列を使用する際の注意点}
        可変長配列「vector」を用いる場合には,
        ヘッダーファイルにvectorを追加しなければならない.\footnote{ヘッダーファイルがbits/stdc++.hなら不要.}
    \end{alertblock}
\end{frame}

\begin{frame}{vectorの宣言}
    vectorの宣言は通常の配列とは異なるので注意する必要がある.
    \begin{block}{vectorの宣言}
    vector<型> 変数名(最初の要素数);

    例として, vector<int> a(10);
    \end{block}
\end{frame}

\begin{frame}{vectorの初期化}
    vectorの初期化は固定長配列と同様に行ってもよいが,すべての要素に同じ値を代入したい場合は本項のやり方がおすすめである.
    \begin{block}{vectorの初期化}
        vector<型> 配列名(要素数,初期値);

        例として,vector<int> a(10, 1\footnote{a[0] $\sim$ a[9]までに1が代入される.});
    \end{block}
    この方法を用いることでfor文を回すことなく配列を初期化することができる.
    
\end{frame}

\begin{frame}{vectorの使い方}
    vectorの使い方についてはC++のリファレンスを参照していただきたい.
    \url{https://cpprefjp.github.io/reference/vector/vector.html}
    
    以下は,vectorを扱う際によく用いるものの代表例を挙げている.
    \begin{table}[h]
        \centering
        \begin{tabular}{|l|l|}
            \hline
            名前 & 説明\\
            \hline
            push\_back & 末尾へ要素を追加\\
            pop\_back & 末尾から要素を削除\\
            insert & 要素の挿入\\
            \hline
        \end{tabular}
        \caption{コンテナの変更}
        \label{tab:my_label}
    \end{table}
\end{frame}
\end{document}